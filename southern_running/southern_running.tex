\documentclass[14pt, oneside]{memoir}

\title{Southern Running}
\author{Walt Askew}
\date{\today}

\begin{document}
\maketitle

I'm surprised more great distance runners don't come out of the
South.
I can only guess that potential runners are lost to football or other
sports.
Because as perfect as Southern culture seems for breeding distance
runners, I'd expect the Olympic Committee would have had to force the
South to start fielding its own team to keep the marathon
competitive.

Running is the sport of accepting suffering.
'Sport' is even too strong a word for it.
Skill, technique and strategy are the sort of things you need for
actual sports.
Running just requires a body that can put one foot in front of the
other and a head that won't ask it stop.

The Southern obsession with suffering has become internationally
visible as grievance politics have become the animating engine of the
Republican party.
The same folks who have claimed victimhood under the 'War of Northern
Aggression,' carpetbaggers, the existence of a federal government and
racial minorities seeking dignity are now seeking government
protection from the ever-present bogeyman of oppressive woke-ness.
But the obsession goes much deeper.
It's there with every preacher reminding their congregation that their
reward lies not within this world but the next and every coach telling
their students to suck it up after a concussion.
It's there most poetically in Flannery O'Connor's Hazel Motes,
stuffing his shoes with gravel to walk miles in penance to a god he
aches not to believe in.

The world isn't something to be enjoyed.
It's something to suffer through.
Every time I lace up my shoes and begin stretching before I run, I do
so expecting to suffer.
In running at least this is a productive attitude.
Running is terrible.

My high school cross country couch used to relate a potentially
aprochyphal anecdote about world class marathon runner.
An interviewer asked him how long into the race it took before the
pain started to set in.
``Fifteen minutes,'' he replied.
A fifteen minute prelude to over two hours of pain.
This was one of many ways our coach explained to us that running is
supposed to hurt.
Folks might imagine that with enough training running starts to get
easier.
That ragged breaths will smooth and the fires which start smoldering
from the top of the thighs and spread from there will eventually go
out.
The truth is the opposite.
The truth is that the training is for weathering pain, not
extinguishing it.

For my coach, a bad race was a moral failing much more than a physical
one, and his coaching descended from that principle.
``Askew!'' he'd shout as he ran the warmup together with the team.
``Yes coach?'' I'd reply.

``Askew --- did Darwin make the sun?''

``The sun?''

``That's right, the sun. Darwin, Charles Darwin --- did he make the
sun?''

``No, coach.'' We'd been over this before.
Charles Darwin, who did not create the sun, although many people seem
to act like he did, had in fact created an unscientific,
unfalsifiable and unproven theory on the origin of life,
as my coach would explain once more.
``That's right, Askew. God --- God created the sun.''

Across warmups, cooldowns and long, slow distance runs, essentially
any time he had the breath to spare, coach took advantage of his
captive audience.
We learned that American had in fact won Vietnam (he was, for better
or worse, a history teacher), modern art is liberal propaganda, and
the many sacrileges of communism.
Our practices took on historical proportions.
``Y'all look like the French after Verdun!'' he'd shout after a
disappointing practice, ``The French \textit{lost} Verdun!''
On a good day, we were Americans storming Normandy.

It's important to understand that coach wasn't just a crank performing
for an impressionable audience or a zealous street preacher (although
he was a bit of those things.)
Moral instruction was a critical part of his coaching strategy.

I'm still running today, the only way I know how.
Seeking out long hills and running up them till I'm ragged and then
down and then up again if I'm not ragged enough.
``Let the hill do the work!'' I can hear my old coach shouting.
I'm still trying to figure out what that one means.
Just with enough self-awareness now that I won't make that click in my
thigh worse.

\end{document}
