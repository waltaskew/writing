\documentclass[12pt, oneside]{memoir}

% maybe change title to avoid over-generalizing about the south
\title{Southern Running}
\author{Walt Askew}
\date{\today}

\begin{document}
\maketitle

I don't think of distance running as a sport.
Actual sports involve skill, technique or strategy.
Maybe there's another level of running where elegant flips of the
ankle or advanced strategies for putting one foot in front of the
other come into play, but I haven't found it yet.
If I try to imagine what color commentary for running-if-it-was-sport
sounds like, the words tumble into farce,
``Look at the way he's making that turn! Just the perfect angle of
attack going into it -- I wonder what's going through his head right
now?''
It's not much of a mystery to me what's going on inside his head right
now. 
It's alternating between a sharp blue exhausted emptiness and
bargaining.
Bargaining between the mind and the body, alternately begging and
cajoling and pleading and prodding, trying to convince the legs to
keep pounding forward.
Promising that the end is in sight.
Just a little longer until it's time to collapse into a tired heap of
sweat and Gatorade. 
Begging, cajoling, pleading and prodding to push through the pain.

I've heard some people profess to enjoy running.
I don't exactly understand what they mean.
When I lace up my shoes and begin stretching for a run, I'm not
getting ready for endorphins and dopamine.
I expect, at some point before I return, to suffer for a bit.
Not nobly or valiantly or meaningfully.
And certainly not suffering comparable to that which people feel from
loss or violence or hardship that isn't a self-imposed jog around a
hilly circle.
But at some point, I know my legs will begin to burn as they turn
soft and my breath will tear sharply into the back of my throat and
still not be enough to fill my ravenous lungs and a voice inside my
head will beg me to stop and the rest of my body which has no voice
will shout wordlessly for me to stop.
And I have to say no.
Just as a matter of course.
This is a suffering I've spent too much of my life preparing to
accept.

My high school cross country coach used to relate a potentially
apocryphal anecdote about a world class marathon runner whose name
I've forgotten.
An interviewer asked how long into the race he got before it started
to hurt.
``Fifteen minutes,'' he replied.
A fifteen minute prelude to over two hours of pain.
Coach was good at setting expectations if nothing else.
I might have naively imagined that with enough training,
running would start to get easier.
That'd I'd hone myself until speed became effortless.
The opposite is true.
Training supplies the tools for pushing a body harder and harder at an
ever more unreasonable pace.
It unlocks ever more impossible lengths to push through.
A never ending parade of barriers glowering on the horizon.

This short anecdote contained our coach's entire coaching strategy.
That running is supposed to hurt, sure.
Rubbery legs or ragged breaths would never be an excuse to skip a lap.
But more fundamentally, that running is as much about mental
fortitude as physical ability.
That a successful runner is one who does not shy from pain but humbly
accepts it.
He was telling us that training is for learning to weather pain, not
extinguishing it.
So he set about training our minds as best he knew how.

``Askew!'' he'd shout as he ran the warm-up with the team.
``Yes, Coach?'' I'd reply.

``Askew, did Darwin make the sun?''

``The sun?''

``That's right, Askew. The sun. Darwin, Charles Darwin -- did he make
the sun?''

``No, Coach.'' We'd been over this before.
Coach explained that Charles Darwin, who did not create the sun,
although many people seemed to act like he did, had in fact created an
unscientific, unfalsifiable and unproven theory on the origin of life.
For him, the idea that men had come from monkeys was degrading,
defeatist.
That man came from God and was destined for greatness was clear, and
those who claimed otherwise were apologists for man's worst instincts.
``That's right, Askew. God. God created the sun.''

It's important to understand that Coach wasn't just a zealous street
preacher or a crank performing for an impressionable young audience
(although he was a bit of both those things.)
Moral instruction was a critical part of his coaching strategy.
It's what would give us the strength to push through the most
agonizing leg of a race.
No monkey would sign up to run in circles to the point of absolute
exhaustion for no reward besides a plastic medal.
Willingly accepting suffering to no ostensible end is the sole
provenance of humans.
Coach saw putting aside the wants of the body for the pursuit of
something greater as the essential matter of running.
No wonder he pulled religion into it.
Fasting, putting on hairshirts and chains, self-flagellation --
Christianity provides an entire vocabulary for mortification of the
flesh, an entire theology of the body as something to be exceeded.
It offers a vision of the world as a toy of the devil, a place where
suffering should be expected as part of the journey from a fallen
world to heaven.

Not many folks in the Bible walk an easy path.
Moses never gets to step foot in the holy land.
Joseph gets tossed in a well and sold into slavery.
Jesus is crucified.
Job's suffering fills an entire book of the Bible, all for a wager
he wasn't even betting on.
But all this suffering has a noble gilt to it.
It's variously cast as deserved penance, self sacrifice, or a measure
of the strength of a faith that never wavers.
Some folks speak about the martyrdom of early Christians in the same
tone they do the fighting in World War II.
With a mix of nostalgia and jealousy.
A twinge of remorse that they didn't have a chance to go through such
suffering themselves, to prove their faith and obedience to God or
country unassailably.
Without such a struggle to join, they're left to create proving
grounds of their own design.

``Y'all look like the French after Ardennes!'' Coach shouted after one
disappointing practice.
``The French \textit{lost} Ardennes!'' he clarified for anyone yet to
take his History class.
On a good day, we were Americans storming Normandy.
The course weaving in and around our school's football
stadium, which Coach measured and remeasured before every race, became
a site of spiritual and moral struggle.
Cross country running has a curious team aspect I've witnessed many times
but have never been able to satisfactorily explain.
Running is an intensely individual, even private, matter.
There's nothing one teammate can do to help another get across the
finish line any faster short of carrying them (which is, in fact,
cause for disqualification.)
But I've seen it time and again.
Fast or slow races, good or bad practices, come in teams.
Something hits the air and every single runner on the team comes in
twenty seconds off where they usually are.
Where they should be.
We really did start to resemble a beaten army.
But where did that failure come from?

The most obvious place to look would be on the physical plane.
Perhaps the one and only thing every member of the team did share was
the same training regimen.
And the same coach.
But pinning the blame on muscles, bones and sinews never seemed to
explain it.
The feeling of failure came from a pit in my stomach, not a tightness
in my thigh.
For our Coach, the issue was clear.
Our failures were moral failures.
And I felt them in my bones.

One day after a strong race, Coach told me, ``Good work,
Askew. Now if you'd just get your politics straight, you could really
be a great runner.''
I didn't know who'd sold me out, or how he'd figured me out.
He'd heard me admit that Darwin didn't create the sun, after all.
But he'd spotted some piece of closeted progressivism he believed
was getting in the way of the crucial matter of running, of removing
expectations of ease or endorphins to confront pain and spring through
it as the Allies had hails of bullets on D-Day. 

His was a conservatism of rugged self-reliance.
Of dogged self-sufficiency.
Whether real or imaginary being completely besides the point.
A conservatism to which statements like ``The government can hand out
checks, but it can't hand out character'' sound profound.
A conservatism which sees acceptance of aid as a moral failing.
As a dereliction of responsibility.
As an inability to face the struggle and triumph that makes up a life.
So across warm ups, cool downs and long, slow distance runs, essentially
any time he had the breath to spare, Coach took advantage of his
chance to preach to a captive audience, trying to provide the moral
instruction he believed would allow us to become good runners.
We learned that modern art is liberal propaganda, socialism a mental
illness, and that America had in fact won the Vietnam war.
Narratives to the contrary were just examples of defeatism or a
pernicious campaign to obscure America's greatness as an offering to
the altar of cultural relativism.

The upside was that strong team performances were moral and spiritual
triumphs to be celebrated.
Our successes infected him at least as much as our failures.
Some days he'd take on a sublimely childlike glee and hunt for
auspicious treasures.
A broken computer tossed in a dumpster.
A cinder block propping open a door.
Half-rotten logs littering the trails we ran.
He'd gallop to the top of the bleachers while we stretched below,
lugging his treasure to the top row.
And he'd toss his new favorite toy over the bleachers onto the
concrete below, galloping back down to retrieve it and back up again
to throw it over and over and over, whooping and cheering and
grinning, watching with joy as it smashed to pieces.
``Bwoooooh! Bwooo, bwooooooh!'' he'd shout like a joyful, broken fire
alarm.
The other coaches turned their heads, hoping that by ignoring the
scene they were avoiding complicity.

They were adults, and as adults had learned to find such raw
expressions of emotion embarrassing.
But to a group of tired, sweaty teenage boys, this was the greatest
reward we could imagine.
We'd run so well that we'd awakened something in our Coach.
Our successes were written in fragments of cinder block joyously
smashed apart by a man, famous for his severity and explosive temper,
whom we'd reduced to a splitting grin, to a whooping, shouting, bleating,
deranged joy, to a raw expression of his pride in us.
He became one of us in those moments.
Here was an adult that, just like us teenage boys, could imagine
nothing more fun than smashing something to pieces.

His conservatism had pieces of that same glee.
Once, I was at his house, and he was excited to show me the fruits of
one of his favorite hobbies.
He had a menagerie of figurines he'd painted and models he'd built.
They were mostly World War II soldiers.
Rifleman, grenadiers, machine gunners, all in dramatic poses, detailed
and dressed to historic accuracy.
And standing at the center of all these heroes was one little figurine.
Clothed immaculately in a neatly tailored navy suit, stood a tiny
Ronald Reagan. 

After a few years on the team I'd become a true believer, more or
less.
The politics I still wasn't sure about, but the rest I'd internalized.
My times kept getting faster and faster.
The team won state championship after championship, extending Coach's
winning streak into double digits.
I was even on some of those championship teams.
And that central lesson, the twin nobility and necessity of suffering,
I felt in my bones.
Pain became a measure of progress, and accepting it a calling.

Now, I should be clear that Coach never encouraged us to run through an
injury.
There was a line between the burn of a tired muscle and the sear of a
torn one he told us never to cross.
But at some point, I lost the clarity to tell the two apart.

The closet thing I've felt to runner's high is pride.
Something I seldom felt in high school.
The pride I took from overcoming the exhausted pleading of my thighs
while I turned them over and over was much more addictive than
dopamine.
When the starting pistol fired, I never felt I was in competition with
anyone besides myself.
But that competition was fierce.
It was a struggle between my baser instincts begging for rest and a
higher calling demanding more from myself, and I took pride in
disciplining my body to accept that calling.
The taste of cooper pennies hitting the back of my throat that I'd
spit out if my lungs didn't have better things to do, the desperate
pitch of my breath as I pulled into the last few hundred meters of the
race, the minor corrections I'd need to force into my stride as it
became jumbled and gangly from exhaustion -- I'd come to know them
all well enough to establish a grim rapport.
But as I found more ways to push beyond more of what I had once
believed were my limits, I could no longer hear my muscles' panicked
demands to slow down. 
Or if I did hear them, I thought they were just another weakness to
extinguish. 

One day a tear in my thigh intruded on the sublimity of my suffering.
My running career ended in my junior year of high school, not too long
after pulling off a sub-ten-minute two mile on the track.
And twenty years later, long after I've given up on help from physical
therapy, sports massages or chiropractory, my right thigh still makes
an audible pop to complain when I've moved it wrong.
And twenty years later, I'm still working through the lessons I
learned from running.

I don't want to set them all aside.
I wouldn't want to become a Yankee, after all.
The ability to set aside the needs and wants of the body in pursuit of
something greater than that jumble of flesh and bones is surely part
of what makes us human and worth celebrating.
And there's a certain expectation of ease, of a lack of friction, that
digital life seems to encourage which I hope to never fully inhabit.
But at the very least, I could probably afford to pick my battles a
bit better.

A few years ago, my dentist gave me some little toothpick-like sticks
as a sort of alternative to flossing for my badly receding gum line.
She instructed me to gently push them into the gaps between my teeth,
up against my gums, just until they started to lightly blanch with
pressure.
As I wiggled them in and out of my teeth, I figured my dentists must
be wrong.
I wasn't pushing them in hard enough.
My gums were barely blanching.
I could scarcely feel anything.
Which must mean it wasn't working.
So I pushed the picks against my gums, in and out, until they bled and
I took the pain as progress.
My dentist wanted to know where on earth I'd gotten that idea from.

I do still run today, the only way I know how, with a body that puts
one foot in front of the other and a head that won't ask it to stop.
I seek out long hills and run up them till I'm ragged and then
down and up again if I'm not ragged enough.
Though not as quickly, and with a better ear for the whispers of my
body and the limits of my creaking thigh.
``Let the hill do the work!'' I hear Coach shout as I coax myself up
yet another San Francisco Hill.
I'm still trying to figure out what that one means.
Twenty years ago, I saw the hill as a wave to crash against, as a
barrier to exceed which might exhaust other runners less able to throw
themselves over it.
Now, I think Coach wanted us to know that the hill would pull us up
over if, if only we picked up our feet and let it.
And at the top of that hill is a private part of myself I'm happy
continues to exist, that is willing to take up a challenge but content
to see it as nothing more.

\end{document}
