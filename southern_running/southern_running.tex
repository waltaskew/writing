\documentclass[12pt, oneside]{memoir}

% maybe change title to avoid over-generalizing about the south
\title{Southern Running}
\author{Walt Askew}
\date{\today}

\begin{document}
\maketitle

I don't think of distance running as a sport.
Skill, technique and strategy are the sort of things needed for
actual sports.
Maybe there's another level of running where elegant flips of the
ankle or advanced strategies for putting one foot in front of the
other come into play, but I haven't found it yet.
If I try to imagine what color commentary for running-if-it-was-sport
would sound like, it quickly becomes farce.
``Look at the way he's making that turn! Just the perfect angle of
attack going into it --- I wonder what's going through his head right
now?''
It's not much of a mystery to me what's going on inside his head right
now. 
It's alternating between a sharp blue exhausted emptiness and
bargaining.
Bargaining between the mind and the body, alternately begging and
cajoling and pleading and prodding, trying to convince the legs to
keep pounding forward.
Promising that the end is in sight.
Just a little longer until it's time to collapse into a tired heap of
sweat and Gatorade. 
Begging, cajoling, pleading and prodding to push through the pain.

When I lace up my shoes and begin stretching for a run, I do so with
trepidation.
I expect, at some point before I return, to suffer for a bit.
Not nobly or valiantly or meaningfully.
And certainly not suffering comparable to that which people feel from
loss or violence or hardship that isn't a self-imposed jog around a
hilly circle.
But at some point, I know my legs will begin to burn as they turn
soft and my breath will tear sharply into the back of my throat and
still not be enough to fill my ravenous lungs and a voice inside my
head will beg me to stop and the rest of my body which has no voice
will shout wordlessly for me to stop.
And I have to say no.
Just as a matter of course.
This is a suffering I've spent too much of my life preparing to
accept.

My high school cross country coach used to relate a potentially
apocryphal anecdote about a world class marathon runner whose name
I've forgotten.
An interviewer asked how long into the race he got before it started
to hurt.
``Fifteen minutes,'' he replied.
A fifteen minute prelude to over two hours of pain.
Coach was good at setting expectations if nothing else.
I might have naively imagined that with enough training,
running would start to get easier.
That'd I'd hone myself until speed became effortless.
The opposite is true.
That strength just becomes fuel for pushing the body harder and harder
at an ever more unreasonable pace.
Training unlocks ever more lengths to push through.

This short anecdote contained our coach's entire coaching strategy.
That running is supposed to hurt, sure.
Rubbery legs or ragged breaths would never be an excuse to skip a lap.
But more fundamentally, that running is as much about mental
fortitude as physical ability.
That a successful runner is one who doesn't shy from pain.
A runner should just accept it, as a matter of course.
He was telling us that training is for learning to weather pain, not
extinguishing it.
So he set about training our minds as best he knew how.

``Askew!'' he'd shout as he ran the warm-up with the team.
``Yes, Coach?'' I'd reply.

``Askew, did Darwin make the sun?''

``The sun?''

``That's right, Askew. The sun. Darwin, Charles Darwin --- did he make
the sun?''

``No, Coach.'' We'd been over this before.
Coach explained that Charles Darwin, who did not create the sun,
although many people seemed to act like he did, had in fact created an
unscientific, unfalsifiable and unproven theory on the origin of life.
For him, the idea the men had come from monkeys was degrading,
defeatist.
That man came from God and was destined for greatness was clear, and
those who claimed otherwise were apologists for man's worst instincts.
``That's right, Askew. God. God created the sun.''

It's important to understand that Coach wasn't just a crank performing
for an impressionable young audience or a zealous street preacher
(although he was a bit of both those things.)
Moral instruction was a critical part of his coaching strategy.
It's what would give us the strength to push through the most
agonizing leg of a race.
No monkey would sign up to run in circles to the point of absolute
exhaustion for no reward besides a plastic medal.
Willingly accepting suffering to no ostensible end is the sole
provenance of humans.
Coach saw putting aside the wants of the body for the pursuit of
something greater as the essential matter of running.
No wonder he pulled religion into it.
Fasting, putting on hairshirts and chains, self-flagellation ---
Christianity provides an entire vocabulary for mortification of the
flesh, an entire theology of the body as something to be exceeded.

Not many folks in the Bible walk an easy path.
Moses never gets to step foot in the holy land.
Joseph gets tossed in a well and sold into slavery.
Jesus is crucified.
Job's suffering fills an entire book of the Bible, all for a wager
he wasn't even betting on.
But all of this suffering has a noble gilt to it.
It's variously cast as deserved penance, self sacrifice, or a measure
of the strength of a faith that never wavers.
Some folks speak about the martyrdom of early Christians in the same
tone they do the fighting in World War II.
With a mix of nostalgia and jealousy.
A twinge of remorse that they didn't have a chance to go through that
suffering themselves, to prove their faith and obedience to God or
country unassailably.
Without such a struggle to join, they're left to create proving
grounds of their own design.

``Y'all look like the French after Ardennes!'' Coach shouted after one
disappointing practice.
``The French \textit{lost} Ardennes!'' he clarified for anyone yet to
take his History class.
On a good day, we were Americans storming Normandy.
The course weaving in and around our school's football
stadium, which Coach measured and remeasured before every race, became
a site of spiritual and moral struggle.
Cross country running has a curious team aspect I witnessed many times
but have never been able to explain.
Running is an intensely individual, even private, matter.
There's nothing one teammate can do to help another get across the
finish line faster short of carrying them (which is, in fact, cause
for disqualification.)
But I've seen it time and again.
Fast or slow races, good or bad practices, come in teams.
Something hits the air and every single runner on the team comes in
twenty seconds off where they usually are.
Where they should be.
We really did start to resemble a beaten army.
But where did that failure come from?

The most obvious place to look would be on the physical plane.
Perhaps the one and only thing every member of the team did share was
the same training regimen.
But putting the blame on the sinews inside my legs never seemed to
explain it.
The feeling of failure came from a pit in my stomach, not a tightness
in my thigh.
For our Coach, the issue was clear.
Our failures were moral failures.

I remember one day after a strong race, he told me, ``Good work,
Askew. Now if you'd just get your politics straight, you could really
be a great runner.''
% explain more what this means about my politics being wrong - what is he worried about
% get a wink to the reader in here
I don't know who sold me out.
Or how he figured me out --- he'd heard me admit that Darwin didn't
create the sun, after all.
But he'd spotted some piece of what I'd hoped was closeted
progressivism that he believed was getting in the way of the crucial
matter of running.
Of stepping back from expectations of ease or endorphins, confronting
pain and sprinting through it as the Allies had hails of bullets on
D-Day.

Across warm ups, cool downs and long, slow distance runs, essentially
any time he had the breath to spare, Coach took advantage of his
chance to preach to a captive audience, trying to provide the moral
instruction he believed would allow us to become good runners.
We learned that American had in fact won the Vietnam war.
(He was, for better or worse, a history teacher at our school.)
Narratives to the contrary were just examples of defeatism or a
pernicious campaign to obscure America's greatness as an offering to
the altar of cultural relativism.
Our practices took on historical proportions.

The upside is that strong team performances were examples of a moral
and spirituality that needed to be celebrated.
The successes infected him at least as much as the failures.
He took on a sublimely childlike glee.
Some days after practice he'd find some bit of detritus.
A broken computer hiding a dumpster.
A cinder block propping a door open.
He'd walk to the top of the bleachers we were all stretching behind,
lugging his bit of treasure to the top row.
And he'd toss his new favorite toy over the bleachers onto the
concrete below, over and over and over, whooping and cheering and
grinning, watching with joy as it smashed to pieces.
% make this more of a scene
% what did the team think here? how did they react? the other coaches?
% describing unhinged behavior in a matter-of-fact way is fun

His conservatism even had a bit of that same glee to it.
Once I was at his house, and he walked me to a back room to show off
the fruits of one of his hobbies.
He had a menagerie of figurines he'd painted and models he'd built.
They were mostly World War II soldiers.
Rifleman, grenadiers, machine gunners, all in dramatic poses, detailed
and dressed to historic accuracy.
And stood at the center of all these heroes was one little figurine.
Clothed immaculately in a neatly tailored navy suit, stood a
tiny Ronald Reagan.

% collection of pieces, suffering as the thing binding them together
% focus on my experience and myself to help avoid over-generalizing
% about the whole South.
% focus more on me, what are the questions I still have
% opportunity to provide some nuance for the south, some compassion
% what questions do I have?
% put myself in here more
% get some more of my reflections in here

% maybe read some david sedaris

% I'm still running today, the only way I know how.
% Seeking out long hills and running up them till I'm ragged and then
% down and then up again if I'm not ragged enough.
% ``Let the hill do the work!'' I can hear Coach shouting.
% I'm still trying to figure out what that one means.







There's a strain of Christianity which isn't as prominent these days
but winds back two thousand years.
It says the world isn't something to be enjoyed.
The world is something to suffer through.
It's saying, ``Your reward is not in this world, but the next,'' and
actually meaning it.

Flannery O'Connor gave life to it most poetically in the character of
Hazel Motes, stuffing his shoes with gravel to walk miles in bloody
penance to a god he aches not to believe in.
I wouldn't suggest that the South has proven especially resistant to the
currently dominant prosperity gospel strain of Christianity.
But this older understanding of the world as a toy of the devil, a
place where suffering should be expected, just as a matter of course,
as part of the journey from a fallen world to heaven, is still alive
and well in parts of it.

\end{document}
