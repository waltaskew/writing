\documentclass[12pt, oneside]{memoir}

\title{Shattered Animals}
\author{Walt Askew}
\date{\today}

\begin{document}
\maketitle


I live in a world of words. 
My life is spent rearranging them.
I go to work and find problems made of words and I supply the words to
solve them.
I talk with friends about their problems and we reshuffle words until
the problems feel tractable.
I tutor students and we search together for the missing words their
teacher and textbook didn't say to unlock the understanding they're
grasping towards.
When I'm gone, all that will be left of me are the things I used to
say.
Dripping away in memory.

Emails and pep talks and brainstorming and computer code.
Power points and patents and contracts and intellectual property.
Laws and rulings and states of the unions and judicial decrees.
Culture and meaning and philosophy and religion.
We've built so much atop foundations of dictionaries.

But sometimes the realities of having a body collapse. 
Words can’t shake what’s fundamentally a problem in flesh. 
Death and dying, for instance. 
There’s nothing to say.
They just are. 
And they leave us shattered animals. 

Confrontation with these problems of the flesh is uncomfortable.
At best, embarrassing.
A decaying body.
Being held up at the point of a knife.
Puddles of fear, pangs of hunger, shocks of pain.
There are no words.
We have to no choice but to solve these problems like animals.

I always feel uncomfortable at airport.
I've left my body in the wrong place.
And now I need to pick it up and launch it through the air to drop it
in the right place.
Only to buy a return ticket, because I know it'll still be in the
wrong place once I get there.
It's a purely animal problem.
There aren't words to solve my misplaced body.
I have to join a herd of ticket holding jackals.
We hiss and sigh at every disruption in the every queue.
We mark our territory with flourishes of the elbow over the hand
rests.
We circle the luggage carousel, waiting for a chance to pounce.

Human memories are editorials. 
They are stories we tell ourselves about ourselves. 
We revise them constantly. 
Edit them for clarity and brevity. 
Turn them into a narrative. 
Retell a few old favorites to ourselves to remind ourselves who we
are.
Memories about being a shattered animal are not like this. 
They are also without words. 

I don’t like to think about the time when my closest friend was murdered. 
They’re sad memories.
She's the one person I've been closest to in this world, and I suspect
I'll never be as close to anyone as I was to her again.
But that’s not it. 
These aren’t memories like other memories.
They don’t deserve to share the same word. 
They are unedited films I can’t walk out of the theater from. 
I experience them from beginning to end and without rest. 
It takes time. 
The remembering itself takes time. 

The phone call and the refusal to believe and the desperation to not
believe and the impossibility of sleep. 
The unavoidable confirmation. 
The phone call with a detective who was disappointed and practiced in
politely hiding his impatience at my lack of useful evidence besides a
vague bad feeling while I waited in line for a tasteless torta.
The sorting of all the clothes and books and paints and brushes and
cats that had been hers but were now dead by association.  
They had reached a heightened inertness when she died. 
Clothes with no one to wear them. 
Canvases with no one to paint them. 
The dead leave behind so much potential that becomes detritus. 
The loading of it all into to a truck driven by movers who were kind
enough to appear somber but were ultimately just bored.
The joke I made to my friend about how he had somehow never learned to
pack a cigarette that came out like an accusation because jokes had
stopped working because words had stopped working.
The feeling of unfeeling. 

It all plays from beginning to end like a clacking film to an empty
theater.
Eerily the same each time in a way normal memories never are.
Even my memories have nothing to say.

Her murder left me a shattered animal.
I confronted it wordlessly because there are no words and faced the
animal reality of the violence that killed her.
I still have nothing to say.

There’s something perverse about being expected to learn from this. 
To grow. 
To come back stronger or wiser. 
To end this with a paragraph revealing a moral to the story or lesson
learned or insight into the human condition.

Raw experiences aren't like this. 
They don't order themselves with a beginning, middle and an end
nestled with well built phrases.
They don’t orderly form a line to successively hone one into a better
person.
Only human memories are like this. 
The stories we tell about ourselves to ourselves are like this. 
But real experiences, intrusions of our animal existence, are not. 
They just are. 
They defy language and narrative. 
They leave us shattered animals because they leave us animals. 
Without words to hide behind. 

\end{document}
